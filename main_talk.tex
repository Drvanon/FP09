\documentclass{beamer}
\usepackage[utf8]{inputenc}

%Information to be included in the title page:
\title{Seminar Talk \\ F09: Neuromorphic Computing}
\author{Students: Robin Dorstijn \& Moritz Epping \\ Supervisor: Jakob Kaiser}
\institute{Universität Heidelberg}
\date{January 2021}

\begin{document}

\frame{\titlepage}

\begin{frame}
    %%%
    \frametitle{Summary}
    \begin{enumerate}
        \item Theoretical Overview
        \item Results
        \item Discussion
    \end{enumerate}
    %%%
\end{frame}

\begin{frame}
    %%%
    \frametitle{Motivation}
   	\begin{itemize}
   		\item Goals of Neuromorphic Computing:
   		\begin{enumerate}
   			\item Building more efficient computers (Von-Neumann Bottleneck vs.  Evolution)
   			\item  Understanding brain by trying to rebuild it!	
   		\end{enumerate}
   		\bigskip
   		\item Proposed Architecture: SPIKEY (Heidelberg)
   		\begin{enumerate}
   			\item What are basic properties neuromorphic computing devices must fulfill?
   			\item Does SPIKEY fulfill these properties?
   			\item How well can we use it for more complicated purposes?
   		\end{enumerate}
   	\end{itemize}
\end{frame}

\begin{frame}
    %%%
    \frametitle{Theoretical Overview - Neurons}
    \begin{columns}
          \column{0.28\linewidth}
             \centering
             \includegraphics[height=5cm, width=3.5cm]{figures/neuron_script.png}
           \column{0.68\linewidth}
              \begin{itemize}
   	
   				\item Biological Neuron
   				\begin{itemize}
   					\item Basic computational units of the brain
   					\item Electronic information processing
   					\item Receive input signals (ion waves) via \textbf{dendrites}
   					\item Depending on input: Create a signal
   					\item Transmit signal to other neurons
   				\end{itemize}
   				
   				\item form a network with several interconnected neurons
 			\end{itemize}
	\end{columns} 
   	
\end{frame}

\begin{frame}
    %%%
    
    \frametitle{Theoretical Overview - LiF Model}
    
     \begin{figure}
    		\centering
    		\includegraphics[width=0.8\textwidth]{figures/lif_script.png}
    		%\caption{source: FP09 neuromorphic computing script,  2020}
    \end{figure}
    
     \begin{itemize}
   	
   		\item Electronical Analogy: LIF - Circuit
   		\begin{itemize}
   				\item \textbf{L}eaky \textbf{I}ntegrate and \textbf{F}ire
   				\item Treat neuron as capacitor
   				\item Currents charging / decharging it (input)
   				\item Op-Amp: Compare capacitor voltage to \textit{threshold voltage},  if 
   				$V\geq V_{th}$ then spike
   		\end{itemize}
 	\end{itemize}

\end{frame}

\begin{frame}
	\frametitle{Hardware realization: Spikey}
	\begin{columns}
          \column{0.5\linewidth}
          	\begin{figure}
    				\centering
    				\includegraphics[width=\linewidth]{figures/script_spikey_image.png}
    				%\caption{source: FP09 neuromorphic computing script,  2020}
 		   \end{figure}
 		   \begin{itemize}
          		\item Spikey as a collection of LiF circuits interconnected with synapses
          	\end{itemize}
          
          \column{0.5\linewidth}
          \begin{figure}
    				\centering
    				\includegraphics[width=\linewidth]{figures/script_spikey_schematic.png}
    				%\caption{source: FP09 neuromorphic computing script,  2020}
 		   \end{figure}
 		   \begin{itemize}
          		\item Schematic of the basic operation process
          	\end{itemize}
          	
	\end{columns}
\end{frame}


\begin{frame}
    %%%
    \frametitle{Results - 1: basic firing behaviour} 
    %%%    
    \begin{figure}
    		\centering
    		\includegraphics[width=0.6\linewidth]{figures/fp_task1_1membrane.png}
    \end{figure}
    
    \begin{itemize}
    		\item no excitatory/inhibitory synapses,  just leakage potential,  
    		\textbf{but: } $E_l>V_{th}\quad \Rightarrow$ Spikes
    		\item Firing rate measured to $t_{fir} = (16.10\pm 0.11)$; dependent on 
    		time constant, leakage potential,  leakage conductance
    \end{itemize}
\end{frame}

\begin{frame}
    %%%
    \frametitle{Results - 1: basic firing behaviour} 
    %%%    
\end{frame}

\begin{frame}
    %%%M
    \frametitle{Results - 1: single neuron} 
    %%%   
    
    
\end{frame}

\begin{frame}
    %%%R
    \frametitle{Results - 2: calibrating neuron parameters} 
    %%%    
\end{frame}

\begin{frame}
    %%%M
    \frametitle{Results - 3: single neuron with synaptic input} 
    %%%    
    \begin{itemize}
    		\item we need to evaluate how a neuron responds to synaptic input
    		\item stimulate a neuron with a synapse,  then \textbf{record its membrane 
    		potential} (EPSP).
    		\item Two parameters:
    		\begin{enumerate}
    			\item drvifall: scale magnitude of signal
    			\item drviout: scale decay properties 
		\end{enumerate}  
		\item shape of EPSP depends on \textbf{synapse type}:
		\begin{enumerate}
			\item excitatory synapses: positive spikes
			\item inhibitory synapses: negative spikes
		\end{enumerate}		  		    
    \end{itemize}
\end{frame}

\begin{frame}
    %%%M
    \frametitle{Results - 4: short term plasticity} 
    \begin{itemize}
    		\item \textbf{Plasticity} refers to the ability of a neuron to dynamically 
    		change the synapse weights.
    		\item two types:
    		\begin{enumerate}
    			\item short-term plasticity
    			\item spike timing dependent plasticity
    		\end{enumerate}
    		\item Spikey implements simplified \textbf{Tsodyks Markram Model}.  
    		Neurotransmitters of synapse are in one of three states $(R,E,I)$ with
    		\begin{align}
			1 &= R + E + I  \\
			\frac{dE}{dt} &= - \frac{E}{\tau_\text{facil}} + \sum_\text{spike} 
			UR\delta(t-t_\text{spike} ) \\
			\frac{dR}{dt} &= - \frac{I}{\tau_\text{rec}} - \sum_\text{spike} 
			UR\delta(t-t_\text{spike} ) 
\end{align}
    \end{itemize}
    %%%    
\end{frame}

\begin{frame}
    %%%R
    \frametitle{Results - 5: feed-forward networks} 
    %%%    
\end{frame}

\begin{frame}
    %%%M
    \frametitle{Results - 6: recurrent networks} 
    %%%  
    \begin{columns}
          \column{0.5\linewidth}
          	\begin{figure}
    				\centering
    				\includegraphics[width=\linewidth]{figures/activity_sweep.png}
    				%\caption{source: FP09 neuromorphic computing script,  2020}
 		   \end{figure}
 		   \begin{itemize}
          		\item Spikey as a collection of LiF circuits interconnected with synapses
          	\end{itemize}
          
          \column{0.5\linewidth}
          \begin{figure}
    				\centering
    				\includegraphics[width=\linewidth]{figures/CV_sweep.png}
    				%\caption{source: FP09 neuromorphic computing script,  2020}
 		   \end{figure}
 		   \begin{itemize}
          		\item Schematic of the basic operation process
          	\end{itemize}
          	
	\end{columns}
    
\end{frame}

\begin{frame}
    %%%R
    \frametitle{Results - 7: simple computation XOR} 
    %%%    
\end{frame}

\begin{frame}
    %%%    
    \frametitle{Discussion}
    %%%    
\end{frame}

\end{document}

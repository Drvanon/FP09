\documentclass{beamer}
\usepackage[utf8]{inputenc}
%\usetheme{Boadilla}

%Information to be included in the title page:
\title{Seminar Talk \\ F10: Neuromorphic Computing\\
- Review of the Spikey Hardware - }
\author{Robin Dorstijn \& Moritz Epping \\ Supervisor: Jakob Kaiser}
\institute{Universität Heidelberg}
\date{January 2021}

\usepackage{import}
\usepackage{xifthen}
\usepackage{pdfpages}
\usepackage{transparent}
\usepackage{titlesec}
\usepackage{circuitikz}
\usepackage{physics}



\newcommand{\incfig}[1]{
    \def\svgwidth{\columnwidth}
    \import{./figures/}{#1.pdf_tex}
}

\begin{document}
\expandafter\def\expandafter\insertshorttitle\expandafter{%
  \insertshorttitle\hfill%
  \insertframenumber\,/\,\inserttotalframenumber}


\frame{\titlepage}

\begin{frame}
    %%%
    \frametitle{Summary}
    \begin{enumerate}
        \item Fundamentals
        \begin{itemize}
        		\item Definition / Motivation
        		\item Neurons
        		\item LiF Circuit
        		\item Hardware Realisation
        \end{itemize}
        \item Experiments
        \begin{itemize}
        		\item Fundamental Circuit Behaviour
        		\begin{itemize}
        			\item Firing Behaviour,  Calibration,  Synapse Types,  Plasticity
        		\end{itemize}
        		\item Networks
        		\begin{itemize}
        			\item Feed-Forward,  Recurrent,  XOR-Gate
        		\end{itemize}
        \end{itemize}
        \item Discussion / Scope of Usability
    \end{enumerate}
    %%%
\end{frame}

\begin{frame}
    %%%
    \frametitle{Definition: Motivation}
   	\begin{itemize}
   		\item \textbf{Neuromorphic Computing:} Use of VLSI systems containing
   		electronical analog circuits to mimic architectures from nervous system.
   		\item Goals of Neuromorphic Computing:
   		\begin{enumerate}
   			\item Building more efficient computers (Von-Neumann Bottleneck vs.  Evolution)
   			\item  Understanding brain by trying to rebuild it!
   		\end{enumerate}
   		\bigskip
   		\item Proposed Architecture: SPIKEY (Heidelberg)
   		\begin{enumerate}
   			\item What are basic properties for neuromorphic computing devices?
   			\item Does SPIKEY fulfill these?
   			\item How well can we use it for more complicated purposes?
   		\end{enumerate}
   	\end{itemize}
\end{frame}

\begin{frame}
    %%%
    \frametitle{Theoretical Overview - Neurons}
    \begin{columns}
          \column{0.28\linewidth}
             \centering
             \includegraphics[height=5cm, width=3.5cm]{figures/neuron_script.png}
           \column{0.68\linewidth}
              \begin{itemize}

   				\item Biological Neuron
   				\begin{itemize}
   					\item Basic computational units of the brain
   					\item Electronic information processing
   					\item Receive input signals (ion waves) via \textbf{dendrites}
   					\item Depending on input: Create a signal
   					\item Transmit signal to other neurons
   				\end{itemize}

   				\item form a network with several interconnected neurons
 			\end{itemize}
	\end{columns}

\end{frame}

\begin{frame}
    %%%

    \frametitle{Theoretical Overview - LiF Model}

     \begin{figure}
    		\centering
    		\includegraphics[width=0.8\textwidth]{figures/lif_script.png}
    		%\caption{source: FP09 neuromorphic computing script,  2020}
    \end{figure}

     \begin{itemize}

   		\item Electronical Analogy: LIF - Circuit
   		\begin{itemize}
   				\item \textbf{L}eaky \textbf{I}ntegrate and \textbf{F}ire
   				\item Treat neuron as capacitor
   				\item Currents charging / decharging it (input)
   				\item Op-Amp: Compare capacitor voltage to \textit{threshold voltage},  if
   				$V\geq V_{th}$ then spike
   		\end{itemize}
 	\end{itemize}

\end{frame}

\begin{frame}
	\frametitle{Hardware realization: Spikey}
	\begin{columns}
          \column{0.5\linewidth}
          	\begin{figure}
    				\centering
    				\includegraphics[width=\linewidth]{figures/script_spikey_image.png}
    				%\caption{source: FP09 neuromorphic computing script,  2020}
 		   \end{figure}
 		   \begin{itemize}
          		\item Spikey as a collection of LiF circuits interconnected with synapses
          	\end{itemize}

          \column{0.5\linewidth}
          \begin{figure}
    				\centering
    				\includegraphics[width=\linewidth]{figures/script_spikey_schematic.png}
    				%\caption{source: FP09 neuromorphic computing script,  2020}
 		   \end{figure}
 		   \begin{itemize}
          		\item Schematic of the basic operation process
          	\end{itemize}

	\end{columns}
\end{frame}

\begin{frame}
    %%%
    \frametitle{Results - 1: basic firing behaviour}
    %%%
\begin{figure}[ht]
    \centering
    \begin{circuitikz}[scale = .6, transform shape]
        \draw (4, 0)    to node[midway, above]{$V_m$} (8.5, 0) node[op amp, anchor=+](A1){}; % main line
        \draw (A1.out)  to[short, l=$V_\text{out}$, -o] ++(1, 0);
        \draw (7, 1)    node[left, above] {$V_\text{thres}$} to[short, *-] (A1.-);

        \draw (4, 0)    to[vR, l=$g_l$, *-] (4, -2)
                        to[battery1, l=$E_l$] (4, -3) node[ground] {} (4, -4);
        \draw (6, 0)    to[C, l=$C_m$] (6, -2)
                        node[ground] {} (6, -2);
        \draw (7.5, -2) node[anchor=north] {$V_\text{reset}$}
                        to[short, o-] (7.5, -1)
                        to[normal open switch, -*] (7.5, 0);
        \draw [densely dashed]
              (3.3, -4) --++(0, 4.6)
                        --++(5, 0)
                        --++(0, -4.6)
                        --++(-5, 0);
    \end{circuitikz}
\end{figure}

    \begin{itemize}
    		\item No input from other neurons; just leakage potential
    		\item $E_l>V_{th}\quad \Rightarrow$ regular spiking
    		\item Parameters that influence the spiking rate: $E_l,  g_l,  C_m,
    		t_\text{refrac},  \abs{V_\text{thres} - V_\text{rest}}$.  Recall:
    		\[ \frac{dV(t)}{dt} = \frac{I(t)}{C_m} = \frac{g_l(t)E_l}{C_m} \]
    \end{itemize}

\end{frame}


\begin{frame}
    %%%
    \frametitle{Results - 1: basic firing behaviour}
    %%%
    \begin{figure}
    		\centering
    		\includegraphics[width=0.6\linewidth]{figures/fp_task1_1membrane.png}
    \end{figure}

    \begin{itemize}
    		\item no excitatory/inhibitory synapses,  just leakage potential,
    		\textbf{but: } $E_l>V_{th}\quad \Rightarrow$ Spikes
    		\item Firing rate measured to $t_{fir} = (16.10\pm 0.11)$
    \end{itemize}
\end{frame}

\begin{frame}
    %%%R
    \frametitle{Results - 2: calibrating neuron parameters}
    %%%
    \begin{figure}
        \includegraphics[width=.7\textwidth]{figures/4membranes.png}
        \caption{4 uncalibrated neurons firing at natural time constants.}
    \end{figure}
    \begin{itemize}
        \item Uncalibrated neurons differ in time constants
        \item Calibration required before effective running
    \end{itemize}
\end{frame}

\begin{frame}
    %%%R
    \frametitle{Results - 2: calibrating neuron parameters}
    Suggested algorithm:
    \begin{enumerate}
        \item Set the $g_l$ to the default value and measure the rate of the neuron.
        \item Set the $g_l$ to double the default value and measure the rate of the
            neuron.
        \item Describe the response of the neuron linearly using the last two
            points and estimate where the desired rate would lie.
        \item Set the $g_l$ to this value and measure the rate, if it is within the
        standard error: success! If not return to step 3.
    \end{enumerate}
\end{frame}

\begin{frame}
    %%%R
    \frametitle{Results - 2: calibrating neuron parameters}
    \begin{figure}
        \incfig{algo-base}
    \end{figure}
\end{frame}

\begin{frame}
    %%%R
    \frametitle{Results - 2: calibrating neuron parameters}
    \begin{figure}
        \incfig{algo-1,2}
    \end{figure}
\end{frame}

\begin{frame}
    %%%R
    \frametitle{Results - 2: calibrating neuron parameters}
    \begin{figure}
        \incfig{algo-3}
    \end{figure}
\end{frame}

\begin{frame}
    %%%R
    \frametitle{Results - 2: calibrating neuron parameters}
    \begin{figure}
        \incfig{algo-4}
    \end{figure}
\end{frame}

\begin{frame}
    %%%R
    \frametitle{Results - 2: calibrating neuron parameters}
    \begin{figure}
        \incfig{algo-3-repeat}
    \end{figure}
\end{frame}

\begin{frame}
    %%%M
    \frametitle{Results - 3: single neuron with synaptic input}
    %%%
    \begin{itemize}
    		\item we need to evaluate how a neuron responds to synaptic input
    		\item stimulate a neuron with a synapse,  then \textbf{record its membrane
    		potential} (EPSP).
    		\item Two parameters:
    		\begin{enumerate}
    			\item drvifall: scale magnitude of signal
    			\item drviout: scale decay properties
		\end{enumerate}
		\item shape of EPSP depends on \textbf{synapse type}:
		\begin{enumerate}
			\item excitatory synapses: positive spikes
			\item inhibitory synapses: negative spikes
		\end{enumerate}
    \end{itemize}
\end{frame}

\begin{frame}
	\frametitle{Dependence of the EPSP on the parameters}
	\begin{columns}
          \column{0.4\linewidth}
          	\begin{figure}
    				\centering
    				\includegraphics[width=\linewidth]{figures/epsp_fall_+.pdf}
 		   \end{figure}

          \column{0.4\linewidth}
          \begin{figure}
    				\centering
    				\includegraphics[width=\linewidth]{figures/epsp_fall_-.pdf}
    				%\caption{source: FP09 neuromorphic computing script,  2020}
 		   \end{figure}
	\end{columns}

	\begin{columns}
          \column{0.4\linewidth}
          	\begin{figure}
    				\centering
    				\includegraphics[width=\linewidth]{figures/epsp_out_+.pdf}
 		   \end{figure}


          \column{0.4\linewidth}
          \begin{figure}
    				\centering
    				\includegraphics[width=\linewidth]{figures/epsp_out_-.pdf}
    				%\caption{source: FP09 neuromorphic computing script,  2020}
 		   \end{figure}

          	%% difference: minexc weight mininhweight

	\end{columns}

\end{frame}

\begin{frame}

	\frametitle{Difference between excitatory and inhibitory synapses for same parameter}
	 \begin{columns}
          \column{0.5\linewidth}
          	\begin{figure}
    				\centering
    				\includegraphics[width=\linewidth]{figures/epsp_inh_fall_03_out_05.pdf}
 		   \end{figure}
 		   \begin{itemize}
          		\item Inhibitory Synapse
          	\end{itemize}

          \column{0.5\linewidth}
          \begin{figure}
    				\centering
    				\includegraphics[width=\linewidth]{figures/epsp_exc_fall_03_out_05.pdf}
    				%\caption{source: FP09 neuromorphic computing script,  2020}
 		   \end{figure}
 		   \begin{itemize}
          		\item Excitatory Synapse
          	\end{itemize}

          	%% difference: minexc weight mininhweight

	\end{columns}

	\begin{itemize}
		\item Observe different magnitude despite similar parameters: minExcWeight
		$\neq$ minInhWeight
	\end{itemize}
\end{frame}

\begin{frame}
	%%%M
	\frametitle{Results - 3: noise across different neurons}

	\begin{figure}
		\includegraphics[width=0.7\textwidth]{figures/histo_maxVolt.pdf}
	\end{figure}

	\begin{itemize}
		\item Measured height of EPSP for different triggering excitatory snapses.
		 \item \textbf{Significant variance} between different stimulating neurons.
		  mean: $3.06$ mV,  but asymmetric distribution $\Rightarrow$ median: $2.64$ mV.
		 standard deviation: $1.17$ mV
	\end{itemize}
\end{frame}

\begin{frame}
	\frametitle{Results - 3: Stacked EPSPs}
	\begin{figure}
		\includegraphics[width=0.8\textwidth]{figures/stacked_epsp.pdf}
	\end{figure}
	\begin{itemize}
		\item Stacked EPSPs lead to firing when they are spaced less than $40$ms apart.
	\end{itemize}
\end{frame}

\begin{frame}
    %%%M
    \frametitle{Results - 4: short term plasticity}
    \begin{itemize}
    		\item \textbf{Plasticity} refers to the ability of a neuron to dynamically
    		change the synapse weights.
    		\item two types:
    		\begin{enumerate}
    			\item short-term plasticity
    			\item spike timing dependent plasticity
    		\end{enumerate}
    		\item Spikey implements simplified \textbf{Tsodyks Markram Model}.
    		Neurotransmitters of synapse are in one of three states $(R,E,I)$ with
    		\begin{align}
			1 &= R + E + I  \\
			\frac{dE}{dt} &= - \frac{E}{\tau_\text{facil}} + \sum_\text{spike}
			UR\delta(t-t_\text{spike} ) \\
			\frac{dR}{dt} &= - \frac{I}{\tau_\text{rec}} - \sum_\text{spike}
			UR\delta(t-t_\text{spike} )
\end{align}
    \end{itemize}
    %%%
\end{frame}

\begin{frame}
	\frametitle{Results - 4: Depressing mode}
	\centering
	\includegraphics[width=\textwidth]{figures/stp_depressing.png}
\end{frame}
\begin{frame}
	\frametitle{Results - 4: STP disabled}
	\centering
	\includegraphics[width=\textwidth]{figures/stp_off.png}
\end{frame}

\begin{frame}
	\frametitle{Results - 4: Facilitating mode}

\end{frame}

\begin{frame}
    \frametitle{Results - 5: feed-forward networks}
    \begin{itemize}
        \item Population size
        \item Superposition of feed forward chains as model of networks
        \item Inhibitory neurons in chains
    \end{itemize}
\end{frame}

\begin{frame}
    \frametitle{Results - 5: feed-forward networks}
    \includegraphics[width=\textwidth]{figures/feedforward network.png}
\end{frame}

\begin{frame}
    \frametitle{Results - 5: feed-forward networks}
    \includegraphics[width=\textwidth]{figures/ex0_ex.png}
\end{frame}

\begin{frame}
    \frametitle{Results - 5: feed-forward networks}
    \includegraphics[width=\textwidth]{figures/ex0_inh.png}
\end{frame}

\begin{frame}
    \frametitle{Results - 5: feed-forward networks}
    \includegraphics[width=\textwidth]{figures/ex_ex.png}
\end{frame}

\begin{frame}
    \frametitle{Results - 5: feed-forward networks}
    \includegraphics[width=\textwidth]{figures/ex_inh.png}
\end{frame}

\begin{frame}
    \frametitle{Results - 5: feed-forward networks}
    \includegraphics[width=\textwidth]{figures/inh_ex.png}
\end{frame}

\begin{frame}
    \frametitle{Results - 5: feed-forward networks}
    \includegraphics[width=\textwidth]{figures/feedforward signals loop.png}
\end{frame}

\begin{frame}
    \frametitle{Results - 5: feed-forward networks (discussion)}
    \begin{itemize}
        \item No calibration
        \item Depressing mode?
    \end{itemize}
\end{frame}

\begin{frame}
	%%% include other plots as well?
	\frametitle{Results - 6: recurrent networks}
	\centering
	\includegraphics[width=0.7\textwidth]{figures/rate_CV.png}


	\begin{itemize}
		\item Create recurrent network with sparse connections, synapses inhibitory
		\item Set $E_{leak} > E_{thresh}$ to introduce regular spiking; interrupted
		by inhibitory synapses $\Rightarrow$ Randomness
		\item CV antiproportional to firing rate: lower firing rate $Rightarrow$ was
		interrupted more,  more variation
	\end{itemize}

\end{frame}
\begin{frame}
    %%%M
    \frametitle{Results - 6: recurrent networks: Parameter sweep}
    %%% include other plots as well?
    \begin{columns}
          \column{0.5\linewidth}
          	\begin{figure}
    				\centering
    				\includegraphics[width=\linewidth]{figures/activity_sweep.png}
    				%\caption{source: FP09 neuromorphic computing script,  2020}
 		   \end{figure}
 		   \begin{itemize}
          		\item Spikey as a collection of LiF circuits interconnected with synapses
          	\end{itemize}

          \column{0.5\linewidth}
          \begin{figure}
    				\centering
    				\includegraphics[width=\linewidth]{figures/CV_sweep.png}
    				%\caption{source: FP09 neuromorphic computing script,  2020}
 		   \end{figure}
 		   \begin{itemize}
          		\item Schematic of the basic operation process
          	\end{itemize}

	\end{columns}

\end{frame}

\begin{frame}
	\frametitle{Results - 6: Calibrating the Network}
	\begin{itemize}
		\item We previously showed that activity heavily depends on connection weight
		and connection number
		\item We use this knowledge now to bring the network to a constant average
		firing rate
		\item We calibrate a firing rate of $25Hz$ by setting $w=2.0$ and the number
		of inhibitory synapses to $14$.
	\end{itemize}
\end{frame}

\begin{frame}
    \frametitle{Results - 7: simple computation XOR}
    \includegraphics[width=\textwidth]{figures/XOR-suggestion.png}
\end{frame}

\begin{frame}
    \frametitle{Results - 7: simple computation XOR}
    \includegraphics[width=.7\textwidth]{figures/XOR-used.png}
\end{frame}

\begin{frame}
    \frametitle{Results - 7: simple computation XOR}
    \includegraphics[width=\textwidth]{figures/XOR-output.png}
\end{frame}

\begin{frame}
    \frametitle{Results - 7: simple computation XOR (discussion)}
    \begin{itemize}
        \item No calibration performed other than skipping 'bad' neurons
        \item Depressing mode
    \end{itemize}
\end{frame}

\begin{frame}
   \frametitle{Conclusion}
   \begin{itemize}
       \item Possible to run more complex networks
       \item High instability $\rightarrow$ not production worthy
   \end{itemize}
\end{frame}

\begin{frame}
    %%%
    \frametitle{Discussion}
    %%%
    \begin{itemize}
        \item Desirable: measurability of stability
            \begin{itemize}
                \item $\tau(T)$
                \item $T(t)$
            \end{itemize}
        \item Low population sizes in experiment
    \end{itemize}
\end{frame}

\end{document}

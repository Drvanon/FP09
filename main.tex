\documentclass[a4paper]{article}

\usepackage{siunitx}
\usepackage{hyperref}
\usepackage{circuitikz}
\usepackage{amsmath}
\usepackage{amsfonts}
\usepackage{mathtools}
\usepackage{physics}

\usepackage{import}
\usepackage{xifthen}
\usepackage{pdfpages}
\usepackage{transparent}

\newcommand{\incfig}[1]{
    \def\svgwidth{\columnwidth}
    \import{./figures/}{#1.pdf_tex}
}

\title{FP09 - Neuromorphic computing}
\author{R. Dorstijn (sd249) \& Moritz Epping (hh234)}
\date{December 2020}

\begin{document}
\maketitle
\clearpage
\tableofcontents
\clearpage

\section{Introduction}
\subsection{General theoretical background}
\subsubsection{Motivation}
Brains and computers often are compared because of their shared purpose,
decision making, though they have irreconcilable differences in architecture.
Modern computers function on the basis of the van Neumann
architecture\cite{von-Neumann}, which separates the control unit from the memory
and the computation unit. Though it simplifies the structure of the computer and
makes it modular, it does create a clear bottleneck at the communication layer,
commonly known as the von Neumann bottleneck. This creates an energy
inefficiency that is absolutely unacceptable for biological systems like that of
humans, who spend around 20\% of its total energy uptake on the brain\cite{metabolic-rates}.
This most likely has provided significant evolutionary pressure for hominids to
optimize metabolic resource usage\cite{seymour2016fossil}. For this reason a new
architecture for computers has been suggested: neuromorphic\footnote{``Neuro''
as in brain, ``-morphic'' as in having the shape of, forming ``having the shape
of a brain''.} computing.

\subsubsection{Signal processing in the brain}
Neurons (see fig \ref{fig:neuron}) are the basic computational components of the
brain, transmitting and morphing signals. Each neuron receives signals from
others at the dendrites, which shift the electric potential at the membrane of
the cell at the site of the synapse\footnote{The connection point of two
dendrites where the axon of one meets the dendrite of the other.}, creating an
ion wave crossing the entire membrane of the neuron, reaching it's axon,
allowing it to pass the signal on to other neurons.

\begin{figure}[ht]
    \centering
    \includegraphics[width=\textwidth]{figures/neuron.jpg}
    \caption{Schematic image of a neuron, showing the dendrites, soma and axon.
    Source: Pennsylvania State University}
    \label{fig:neuron}
\end{figure}

The potential at the neuron membrane as a function of time is plotted in figure
\ref{fig:action-potential}, here it is demonstrated how a potential shift caused
by pre-synaptic\footnote{That is the neuron that passes on a signal to the
post-synaptic one.} neurons initiates a spike in the membrane potential, which
forces a release of ion containing vesicles at the terminal bulb, facilitating
the transfer of the signal by changing the relative potential at the dendrites.

\begin{figure}[ht]
    \centering
    \includegraphics[width=.6\textwidth]{figures/action-potential.png}
    \caption{Formation of an action potential. Source:  \url{https://courses.lumenlearning.com/boundless-biology/chapter/how-neurons-communicate/}}
    \label{fig:action-potential}
\end{figure}

\subsubsection{The LIF model}
In order to emulate the behaviour electronically, a leaky integrate and fire
(LIF) model is used. A single neuron is represented by an electronic circuit
like the one in figure \ref{fig:circuit}. When the chip is not excited, nor
inhibited there is a constant ``leak'' voltage present on the membrane,
representing the biological rest potential of the membrane potential. When the
threshold voltage $V_\text{thres}$, the amplifier functioning as a comparator in
this setting, sends out a $V_{CC}$ signal to a digital unit that processes that
the neuron has fired and is responsible for closing the switch that keeps the
``membrane potential'' $V_m$ at some $V_\text{reset}$ for the refractory
period. This time is meant to mimic the period the voltage at the membrane
increases non-linearly and therefore loses its sensitivity to input. The digital
processing unit is also responsible for sending the signal (regardless of if it
is inhibitory or excitatory) to the neurons that the are connected to one that
just fired.

\begin{figure}[ht]
    \centering
    \begin{circuitikz}
        \draw (0, 0)    to node[midway, above]{$V_m$} (8, 0) node[op amp, anchor=+](A1){}; % main line
        \draw (A1.out)  to[short, l=$V_\text{out}$, -o] ++(1, 0);
        \draw (7, 1)    node[left, above] {$V_\text{thres}$} to[short, *-] (A1.-);
        \draw (0, 0)    to[vR, l=$g_l$, *-] (0, -2)
                        to[battery1, l=$E_l$] (0, -3) node[ground] {} (0, -4);
        \draw (2, 0)    to[vR, l=$g_x$, *-] (2, -2)
                        to[battery1, l=$E_x$] (2, -3) node[ground] {} (2, -4);
        \draw (4, 0)    to[vR, l=$g_i$, *-] (4, -2)
                        to[battery1, l=$E_g$] (4, -3) node[ground] {} (4, -4);
        \draw (6, 0)    to[C, l=$C_m$] (6, -2)
                        node[ground] {} (6, -2);
        \draw (8, -3)   to[short, l_=$V_\text{reset}$, o-] (8, -1)
                        to[normal open switch, -*] (8, 0);
    \end{circuitikz}
    \caption{Circuit of a single neuron in a LIF chip.}
    \label{fig:circuit}
\end{figure}

\subsection{Background for the experiments}
\subsubsection{Investigation of a single neuron}
A single neuron will be investigated, without considering any excitatory or
inhibitory influences. This simplifies the circuit of figure \ref{fig:circuit}
to the diagram in figure \ref{fig:circuit-simplified}. The relevant parameters
that appear from this figure are:

\begin{figure}[hb]
    \centering
    \begin{circuitikz}
        \draw (0, 0)    node[ground] {}
                        to[battery1, l=$E_l$]       ++(0, 2)
                        to[vR, l=$g_l$]             ++(2, 0)
                        to[C, l=$C_m$]              ++(2, 0)
                        to[short]                   ++(0, -1)
                        node[ground] {};
        \draw (-2, 0)   node[below] {$V_\text{reset}$}
                        to[normal open switch, o-]  ++(0, 2)
                        to[short, -*]               ++(2, 0);
    \end{circuitikz}
    \caption{Circuit in figure \ref{fig:circuit} simplified for the scenario
    that no signals are incoming for this particular neuron.}
    \label{fig:circuit-simplified}
\end{figure}

\begin{itemize}
    \item $E_l$ setting the \textit{amount} of potential leaking into the `membrane'.
    \item $g_l$ determining how \textit{quickly} the potential leaks into the `membrane'.
    \item $C_m$ describing how much potential is leaked to ground.
    \item $V_\text{reset}$ prescribing the starting point of a cycle.
\end{itemize}

The last item in this list hints at that we do not yet know when the cycle will
end. In the previous section it was stated that $V_\text{thres}$ determines
this value. With this last variable in place it will now be possible to
reconstruct the entire charging and resetting of the $V_m$, this period will be
called $\tau_m$.

However one item is missing: the refraction period. Recall that this is the
period in which the neuron is `off' and does not take in input. Meaning that
this time should be accounted for when calculating the total period.
\begin{equation}
    \tau = \tau_m + \tau_\text{refrac}
    \label{eq:tau}
\end{equation}

\subsubsection{Calibrating neuron parameters}
In the case of a single unconnected neuron the membrane potential can be
understood with application of the Kirchoff formula to the circuit in figure
\ref{fig:circuit-simplified}.
\[
    C_m \frac{dV_m}{dt} = g_l(E_l - V_m)
\]
Clearly, this is a ordinary first order differential equation and has the
solution
\begin{equation}
    V_m(t) = A \exp(\frac{-g_l}{C_m}t) - E_l
    \label{eq:diff-eq-sol}
\end{equation}
where $A$ is given by the initial condition:
\[
    A = V_m(0) - E_l.
\]

In order to set a neuron to fire at a regular frequency $\tau_m$, we have to take
into account that $V_m$ will be reset to $V_\text{reset}$ when it hits
$V_\text{thres}$. Equation \eqref{eq:diff-eq-sol} can be rephrased in terms of
$t$.
\[
    \tau_m = -\ln(\frac{V_\text{thres} - E_l}{V_\text{reset} - E_l})
    \frac{C_m}{g_l}
\]

The characteristic time constant $\tau_c$ of the system can be found by setting
$V_\text{thres}$ as a function of $E_l$ and $V_\text{reset}$.
\[
    V_\text{thres} = E_l - (E_l - V_\text{reset})\exp(-1)
\]
The resulting $\tau_c$ is then only dependent on $C_m$ and $g_l$:
\begin{align*}
    \tau_m &= -\ln(\frac{E_l - (E_l - (E_l - V_\text{reset})\exp(-1))}{V_\text{reset} - E_l}) \frac{C_m}{g_l}\\
           &= -\ln(\frac{(E_l - V_\text{reset})\exp(-1))}{V_\text{reset} - E_l})\frac{C_m}{g_l} \\
           &= \frac{C_m}{g_l} = \tau_c
\end{align*}
This time constant does not take into account the refractory period which should
adds a fixed contribution. This was mentioned in equation \eqref{eq:tau}, but
becomes relevant again when considering the total time.

This describes a good theory of the neuron, however in reality there are some
serious production artifacts that require calibration.

\subsubsection{A Single Neuron with Synaptic Input}
\subsubsection{Short Term Plasticity}
\subsubsection{Feed-Forward Networks}
One of the simplest neural networks one can consider is the feed-forward
network in which every network is arranged so that it passes on the signal it
receives to

\subsubsection{Recurrent Netoworks}
\subsubsection{A Simple Computation: XOR}

\section{Execution \& Results}
A rather interesting setup was provided by the university of Heidelberg. It
hosted a SPIKEY chip on its grounds, which was physically inaccessible due to
the COVID-19 crisis, therefore it was connected to a job manager that could be
written to by the Jülich Supercomputing Center, which hosted Jupyter notebooks
for this purpose. A SPIKEY chip implements the LIF model physically.

\subsection{Investigation of a single neuron}
% Moritz does this part
\subsubsection{Setup/Settings}
\subsubsection{Results}

\subsection{Calibrating neuron parameters}
\subsubsection{Setup/Settings}
On the SPIKEY chip 4 neurons were isolated programmatically and given the same
parameters:
\begin{itemize}
    \item $V_\text{reset}$: \SI{-80.0}{\milli\volt}
    \item $V_\text{thresh}$: \SI{-55.0}{\milli\volt}
    \item $E_\text{leak}$: \SI{-50.0}{\milli\volt}
    \item $g_\text{leak}$:  \SI{20.0}{\nano\siemens}
\end{itemize}

\subsubsection{Results}
This however resulted in wildly differing frequencies, as becomes visible in
figure \ref{fig:4membranes}. In order to compensate for this $g_l$ was adjusted
individually for all of the membranes so that they were all correct within their
respective standard deviation: \SI{20.1}{\milli\volt}, \SI{55.0}{\milli\volt},
\SI{60.0}{\milli\volt}, \SI{20.5}{\milli\volt}.

\begin{figure}[ht]
    \centering
    \includegraphics[width=\textwidth]{figures/4membranes.png}
    \caption{Biological membrane potential of four membranes with the same
    setting. The red dots indicate when the digital processing units recognizes
    that voltage reaches the threshold. The rates for the signals are
    \SI{1.02(2)e1}{\milli\second}, \SI{2.41(3)e1}{\milli\second},
    \SI{2.36(1)e1}{\milli\second}, \SI{1.15(1)e1}{\milli\second}, in the same
    order that they presented in the image. }
    \label{fig:4membranes}
\end{figure}

The scale of the problem becomes even more clear when considering the that one
half of the SPIKEY chip has a rate distribution like presented in figure
\ref{fig:distribution}, the previous method of trial and error becomes rather
infeasible.

\begin{figure}[ht]
    \centering
    \includegraphics[width=\textwidth]{figures/rate-distribution.png}
    \caption{Distribution of firing rate for the same across one side of the
    chip.}
    \label{fig:distribution}
\end{figure}

Instead an algorithm is suggested that should help find a proper calibration for
all neurons that are able to converge on the desired rate.
\begin{enumerate}
    \item Set the $g_l$ to the default value and measure the rate of the neuron.
    \item Set the $g_l$ to double the default value and measure the rate of the
        neuron.
    \item Describe the response of the neuron linearly using the last two
        points and estimate where the desired rate would lie.
    \item Set the $g_l$ to this value and measure the rate, if it is within the
        standard error: success! If not return to step 3.
\end{enumerate}
Sadly we were unable to implement the algorithm, but we encourage the reader
to.

\subsection{A Single Neuron with Synaptic Input}
\subsection{Short Term Plasticity}
\subsection{Feed-Forward Networks}

\subsubsection{Experimental setup/Settings}

\subsubsection{Results}

\subsection{Recurrent Netoworks}
\subsection{A Simple Computation: XOR}

\bibliography{main}
\bibliographystyle{ieeetr}

\section{Appendix}
% Notebook PDFs here

\end{document}
